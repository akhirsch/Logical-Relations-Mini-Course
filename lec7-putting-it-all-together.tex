\documentclass{notes}
\coursename{Logical Relations Mini-Course}
\lecnumber{5}
\title{Putting It All Together}
\author{Andrew K. Hirsch}

\newcommand{\unit}{()}
\newcommand{\Unit}{\texttt{unit}}

\newcommand{\fun}[2]{\lambda\,#1.\;#2}
\newcommand{\app}[2]{#1\;#2}
\newcommand{\inl}[1]{\textsf{inl}(#1)}
\newcommand{\inr}[1]{\textsf{inr}(#1)}
\newcommand{\case}[5]{\textsf{case}(#1, #2.#3, #4.#5)}
\newcommand{\new}[1]{\textsf{new}(#1)}
\newcommand{\deref}[1]{\ensuremath{\mathord{!}e}}
\newcommand{\assn}[2]{#1 \mathrel{\mathord{:}\!\mathord{=}} #2}
\newcommand{\ptr}[1]{\textsf{ref}\;#1}
\newcommand{\pc}{\textit{pc}}
\newcommand{\dom}{\textrm{dom}}

\newcommand{\V}{\mathcal{V}}
\newcommand{\E}{\mathcal{E}}

\newcommand{\bequiv}{\equiv_{\beta}}

\begin{document}
\maketitle

In this lecture, I'm going to look at the language FG (based on the work of Rajani and Garg).
I've simplified the language somewhat, but I've kept in enough complexity to force use to use several ideas.

\begin{syntax}
  \category[Expressions]{e}
    \alternative{x} \alternative{\unit} \alternative{\fun{x}{e}} \alternative{\app{e_1}{e_2}} \\
    \alternative{\inl{e}} \alternative{\inr{e}} \alternative{\case{e_1}{x}{e_2}{y}{e_3}}\\
    \alternative{a} \alternative{\new{e}} \alternative{\deref{e}} \alternative{\assn{e_1}{e_2}}\\
  \category[(Labeled) Types]{\tau, \sigma} \alternative{A^\ell}
  \category[Unlabeled Types]{A} \alternative{\Unit} \alternative{\tau \xrightarrow{\ell} \sigma} \alternative{\tau + \sigma} \alternative{\ptr{\tau}}
\end{syntax}

We say that a type $\tau$ is \emph{protected at} $\ell$ and write $\tau \searrow \ell$ if $\tau = A^{\ell'}$ and $\ell \sqsubseteq \ell'$.


\section{Goal}
\label{sec:goal}

We want to eventually prove the following theorem:
\begin{thm}[Noninterference]
  For any program $e$ such that $x : A^{\ell'} \diamond \pc \vdash e : (\Unit^\bot + \Unit^\bot)^\ell$, where $\ell' \not\sqsubseteq \ell$, and for any $v_1, v_2$ such that $\cdot \diamond \pc \vdash v_i : A^{\ell'}$, if both $e[x \mapsto v_1]$ and $e[x \mapsto v_2]$ terminate, than the produce the same value.
\end{thm}

In the face of effects, this gets harder.
For instance, this program causes problems:
$$\textsf{let}\; x = \new{0} \mathrel{\textsf{in}} \textsf{let}\; \_ = \case{s}{z_1}{\assn{x}{1}}{z_2}{\assn{x}{2}} \mathrel{\textsf{in}} \deref{x} = 1$$
Now, we've leaked the value of the secret $b$, despite the fact that we don't ever directly use that value.

In order to combat this, we add a special label $\pc$ to our judgment.
This is just an information-flow label; however, its job is to constrain the effects in a program based on the control flow of the program that allowed us to get here.
As you can see in the \textsc{case} rule below, when we look at the branches of a \textsf{case} expression we increase the $\pc$.
When we assign to a reference, we ensure that the type we are assigning protects the $\pc$; thus, only high-security variables can be assigned after branching on high-security data.
This would either prevent the assignment above, or it would prevent dereferencing the reference in order to compute a low-security result.

Any proof of nonintereference is going to rely on the $\pc$ to do its job.
We generally prove a separate lemma, known as \emph{containment}, which shows that the \pc is doing its job.
In our case, the containment lemma is going to take the form of the fundamental theorem of a \emph{unary} relation.
We will then define a \emph{binary} relation to express indistinguishability; however, this binary relation will appeal to the unary relation to ensure that the \pc is indeed doing its job.

\section{Typing Rules}
\label{sec:typing-rules}

The judgment $\Gamma \diamond \pc \vdash e : \tau$ means that $e$ has type $\tau$ when $\Gamma$ describes the variables.

\begin{mathpar}
  \infer*[left=var]{ x : \tau \in \Gamma }{\Gamma \diamond \pc \vdash x : \tau}\and
  \infer*[left=lam]{\Gamma, x : \tau \diamond \pc' \vdash e : \sigma}{\Gamma \diamond \pc \vdash \fun{x}{e} : \tau \xrightarrow{\pc'} \sigma} \and
  \infer*[left=app]{\Gamma \diamond \pc \vdash e_1 : (\tau \xrightarrow{\pc'} \sigma)^\ell\\ \Gamma \diamond \pc \vdash e_2 : \tau\\ \sigma \searrow \ell\\ \pc \sqsubseteq \pc'\\ \ell \sqsubseteq \pc'}{\Gamma \diamond \pc \vdash \app{e_1}{e_2} : \sigma} \\
  \infer*[left=inl]{\Gamma \diamond \pc \vdash e : \tau}{\Gamma \diamond \pc \vdash \inl{e} : (\tau + \sigma)^\ell}\and
  \infer*[left=inr]{\Gamma \diamond \pc \vdash e : \sigma}{\Gamma \diamond \pc \vdash \inr{e} : (\tau + \sigma)^\ell}\and
  \infer*[left=case]{\Gamma \diamond \pc \vdash e_1 : (\tau + \sigma)^\ell\\ \Gamma, x : \tau \diamond \pc \sqcup \ell \vdash e_2 : \rho\\ \Gamma, y : \sigma \diamond \pc \sqcup \ell \vdash e_3 : \rho\\ \rho \searrow \ell}{\Gamma \diamond \pc \vdash \case{e_1}{x}{e_2}{y}{e_3} : \rho} \and
  \infer*[left=new]{\Gamma \diamond \pc \vdash e : \tau}{\Gamma \diamond \pc \vdash \new{e} : (\ptr{\tau})^\ell}\and
  \infer*[left=deref]{\Gamma \diamond \pc \vdash e : (\ptr{\tau})^\ell\\ \tau \searrow \ell}{\Gamma \diamond \pc \vdash \deref{e} : \tau} \and
  \infer*[left=assign]{\Gamma \diamond \pc \vdash e_1 : (\ptr{\tau})^\ell\\ \Gamma \diamond \vdash \pc \vdash e_2 : \tau\\ \tau \searrow \pc \sqcup \ell}{\Gamma \diamond \pc \vdash \assn{e_1}{e_2} : \Unit}
\end{mathpar}

\section{Unary Relation}
\label{sec:unary-relation}

Unlike last time, we take our worlds to be maps from pointers to types.
We then build our unary relation mutually-recursively with a semantic typing relation for heaps.
This allows a somewhat more beautiful definition.

We also use step indexing in our model.
This not only helps us when defining semantic typing for heaps, but it also gives us something to found our recursion on when proving the fundamental theorem.

$$
\begin{array}{r@{~\triangleq~}l}
  \mathcal{V}^u(\Unit \mathrel{@} n)(v \mathrel{@} \theta ) & v = \unit\\
  \mathcal{V}^u(\tau \xrightarrow{\pc} \sigma \mathrel{@} n)(v \mathrel{@} \theta) & \exists x, e. v = \fun{x}{e} \land \forall \theta' \succeq \theta, m < n, v.\; \mathcal{V}(\tau \mathrel{@} m)(v \mathrel{@} \theta') \Rightarrow \mathcal{E}_\pc(\sigma \mathrel{@} m)(e[x \mapsto v] \mathrel{@} \theta')\\
  \mathcal{V}^u(\tau + \sigma \mathrel{@} n)(v \mathrel{@} \theta) & \exists v'.\; (v = \inl{v'} \land \mathcal{V}(\tau \mathrel{@} n)(v')) \lor (v = \inr{v'} \land \mathcal{V}(\sigma \mathrel{@} n)(v'))\\
  \mathcal{V}^u(\ptr{\tau} \mathrel{@} n)(v \mathrel{@} \theta) & \exists a. v = a \land \theta(a) = \tau\\
  \mathcal{V}^u(A^\ell \mathrel{@} n)(v \mathrel{@} \theta) & \mathcal{V}(A \mathrel{@} n)(v \mathrel{@} \theta)\\
  \mathcal{E}^u_\pc(\tau \mathrel{@} n)(e \mathrel{@} \theta) &
                                                  \begin{array}[t]{l}
                                                    \forall H. \theta \mathrel{@} n \vDash H \\
                                                    \land \forall m < n. (H, e) \Downarrow (H', v') \Rightarrow\\
                                                    \exists \theta' \succeq \theta. \mathcal{V}(\tau \mathrel{@} m)(v' \mathrel{@} \theta') \land\\
                                                    (\forall a. H(a) \neq H'(a) \Rightarrow \exists \ell'. \theta(a) = A^{\ell'} \land \pc \sqsubseteq \ell') \land\\
                                                    (\forall a \in \mathrm{dom}(\theta') \setminus \mathrm{dom}(\theta). \theta'(a) \searrow \pc)
                                                  \end{array}\\
  \theta \mathrel{@} n \vDash^u  H & \dom(\theta) \subseteq \dom(H) \land \forall a \in\dom(\theta).\; \mathcal{V}(\theta(a) \mathrel{@} n - 1)(H(a) \mathrel{@} \theta)
\end{array}
$$


\section{Binary Relation}
\label{sec:binary-relation}

Here, each world is a triple $W = \langle \theta_1, \theta_2, \beta \rangle$, where each $\theta_i$ maps locations to types, and $\beta$ is a \emph{partial bijection} between $\dom(\theta_1)$ and $\dom(\theta_2)$.
That is, $\beta$ witnesses that some of the locations in $\theta_1$ and $\theta_2$ are supposed to correspond with each other, but not all.
To see why, consider the program
$$\textsf{let}\; x = \new{0} \mathrel{\textsf{in}} \case{b}{\_}{x}{\_}{\new{1}}$$
Here, we may either allocate a new reference or not depending on the high-security input $b$.
Thus, we need to allow one side of the world to grow, but not the other.

$$
\begin{array}{r@{~\triangleq~}l}
  \mathcal{V}^b_{\mathcal{A}}(\Unit \mathrel{@} n)(v \sim v' \mathrel{@} W) & v = v' = \unit\\
  \mathcal{V}^b_{\mathcal{A}}(\tau \xrightarrow{\pc} \sigma \mathrel{@} n)(v \sim v' \mathrel{@} W) &
                                                         \begin{array}[t]{l}
                                                           \exists x, y, e_1, e_2. v = \fun{x}{e_1} \land v' = \fun{y}{e_2} \land\\
                                                           (\forall W' \succeq W, m < n, v_1, v_2. \begin{array}[t]{l}\mathcal{V}_{\mathcal{A}}^b(\tau \mathrel{@} m)(v_1 \sim v_2 \mathrel{@} W') \Rightarrow\\ \mathcal{E}_{\mathcal{A}}^b(\sigma \mathrel{@} m)(e_1[x \mapsto v_1] \sim e_2[y \mapsto v_2] \mathrel{@} W')) \land\end{array} \\
                                                           (\forall \theta \succeq W.\theta_1, m, v.\; \mathcal{V}^u(\tau \mathrel{@} m)(v \mathrel{@} \theta) \Rightarrow \mathcal{E}_\pc^u(\sigma \mathrel{@} m)(e_1[x \mapsto v] \mathrel{@} \theta)) \land\\
                                                           (\forall \theta \succeq W.\theta_2, m, v.\; \mathcal{V}^u(\tau \mathrel{@} m)(v \mathrel{@} \theta) \Rightarrow \mathcal{E}_\pc^u(\sigma \mathrel{@} m)(e_2[y \mapsto v] \mathrel{@} \theta))\\
                                                         \end{array}\\
  \mathcal{V}^b_{\mathcal{A}}(\tau + \sigma \mathrel{@} n)(v \sim v' \mathrel{@} W) &
                                         \exists v_1, v_2.\; (v = \inl{v_1} \land v' = \inl{v_2} \land \mathcal{V}_{\mathcal{A}}^b(\tau \mathrel{@} n) (v_1 \sim v_2 \mathrel{@} W)) \lor (v = \inr{v_1} \land v' = \inr{v_2} \land \mathcal{V}_{\mathcal{A}}^b(\sigma \mathrel{@} n)(v_1 \sim v_2 \mathrel{@} W))\\
  \mathcal{V}^b_{\mathcal{A}}(\ptr{\tau} \mathrel{@} n)(v \sim v' \mathrel{@} W) &
                                           \exists a, a'. v = a \land v' = a' \land a \mathrel{W.\beta} a' \land W.\theta_1(a) = \tau \land W.\theta_2(a') = \tau\\[1em]
  \mathcal{V}_{\mathcal{A}}^b(A^\ell \mathrel{@} n)(v \sim v' \mathrel{@} W) & \left\{\begin{array}{ll} \mathcal{V}^b_{\mathcal{A}}(A \mathrel{@} n)(v \sim v' \mathrel{@} W) & \text{if}~\ell \sqsubseteq \mathcal{A}\\ \forall i \in \{1, 2\}. \forall m. \mathcal{V}^u(A \mathrel{@} m)(v_i \mathrel{@} W.\theta_i) & \text{otherwise} \end{array}\right.\\[3em]
  \mathcal{E}^b_{\mathcal{A}} &
                                \begin{array}[t]{l}
                                  \forall H_1, H_2, m < n. W \mathrel{@} n \vDash H_1 \sim H_2 \land (H_1, e_1) \Downarrow (H_1', v_1)  \land (H_2, e_2) \Downarrow (H_2', v_2) \Rightarrow \\
                                  \exists W' \succeq W. W' \mathrel{@} m \vDash_{\mathcal{A}} H_1' \sim H_2' \land \mathcal{V}^b(\tau \mathrel{@} m)(v_1 \sim v_2 \mathrel{@} W')
                                \end{array}\\[3em]
  W \mathrel{@} n \vDash_{\mathcal{A}} H_1 \sim H_2 &
                      \begin{array}[t]{l}
                        \dom(W.\theta_1) \subseteq \dom(H_1) \land \dom(W.\theta_2) \subseteq \dom(H_2) \land \\
                        \forall (a_1, a_2) \in W.\beta.\; W.\theta_1(a_1) = W.\theta_2(a_2) \land\\
                        \mathcal{V}^b_{\mathcal{A}}(W.\theta_1(a_1) \mathrel{@} n - 1)(H_1(a_1) \sim H_2(a_2) \mathrel{@} W) \land\\
                        \forall i \in \{1,2\}.\; \forall m.\; \forall a_i \in \dom(W.\theta_i).\; \mathcal{V}^u(W.\theta_i(a_i) \mathrel{@} m)(H_i(a_i) \mathrel{@} W.\theta_i)
                      \end{array}
\end{array}
$$


\end{document}

%%% Local Variables:
%%% mode: latex
%%% TeX-master: t
%%% End: