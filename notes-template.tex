\documentclass{notes}
\coursename{Logical Relations Mini-Course}
\lecnumber{1}
\title{Notes Template}
\author{Scribe Name}

\begin{document}
\maketitle

\begin{syntax}
  \category[Expressions]{e}
  % Each alternative is introduced on its own line.
  \alternative{x}
  \alternative{n}
  \alternative{\lambda x . e}
  \alternative{e_1 e_2}

  \category[Values]{v}
  \alternative{\lambda x . e}

  % You can add vertical spacing between categories to visually group them.
  \separate
  % You can pass the amount of space explicitly if you want to manually control it:
  % \separate[5ex]

  \category[Types]{\tau}
  \alternative{\mathbf{Nat}}
  % You can place alternatives on a new line.
  \\
  \alternative{\tau_1 \rightarrow \tau_2}

  \category[Contexts]{\Gamma}
  \alternative{x_1 : \tau_1, \ldots, x_n : \tau_n}
  % You can use words to describe more complicated properties of
  % definitions if you don't want to use BNF.
  \note{no repeats}
  \note{unordered}


  % Alternative layout if you want more space.
  \category[Expressions]{e}
  \groupleft{
    \alternative{x}
    \alternative{n}
  }
  \groupleft{
    \alternative{\lambda x . e}
    \alternative{e_1 e_2}
  }
\end{syntax}

\begin{thm}[Optional Theorem Name]
  Theorem Statement
\end{thm}
\begin{proof}
  Proof here.
\end{proof}

\begin{lem}
  Lemma Statement
\end{lem}

\begin{cor}
  Corollary Statement
\end{cor}

\begin{prop}
  Proposition Statement
\end{prop}

\begin{defn}[Word]
  Definition of word.
\end{defn}

\begin{nb}
  Note
\end{nb}

\end{document}