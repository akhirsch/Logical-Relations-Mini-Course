\documentclass{notes}
\coursename{Logical Relations Mini-Course}
\lecnumber{5}
\title{Recursive Types and Step Indexing}
\author{Shruti Gupta}

\newcommand{\unit}{()}
\newcommand{\Unit}{\texttt{unit}}

\newcommand{\V}{\mathcal{V}}
\newcommand{\E}{\mathcal{E}}

\newcommand{\bequiv}{\equiv_{\beta}}

\begin{document}
\maketitle


\section{Recursive types}

The syntax of the language is further extended with two new expression constructs: roll $e$ and unroll $e$. The type level syntax is augmented with a new ``recursive type'', $\mu X.\;\tau$. 
%
\begin{syntax}
  \category[Types]{\tau}
  \alternative{\Unit}
  \alternative{X}
  \alternative{\tau_1 \rightarrow \tau_2}
  \alternative{\tau_1 + \tau_2}
  \alternative{\tau_1 \times \tau_2}
  \alternative{\forall X.\; \tau}
  \alternative{\mu X.\; \tau}

  \category[Expressions]{e}
  \alternative{\unit}
  \alternative{x}
  \alternative{\lambda x.\,e}
  \alternative{e_1 \; e_2}
  \alternative{\Lambda X.\; e}
  \alternative{e[\tau]}
  \alternative{roll\; e}
  \alternative{unroll\; e}
\end{syntax}
%

  
Typing Rules:


\begin{mathpar}
  \infer{\Delta \vdash \Gamma}{\Delta;\Gamma \vdash \unit:\Unit} (\textsc{Unit}) \and
  \infer{\Delta \vdash \Gamma \\ x : \tau \in \Gamma}{\Delta;\Gamma \vdash x : \tau} \quad (\textsc{Axiom}) \and
  \infer{\Delta; \Gamma, x : \tau_1 \vdash e : \tau_2}{\Delta;\Gamma \vdash \lambda x:\tau_1.\, e : \tau_1 \rightarrow \tau_2} \quad (\rightarrow_I) \and
  \infer{\Delta;\Gamma \vdash e_1 : \tau_1 \to \tau_2 \\ \Delta;\Gamma \vdash e_2 : \tau_1}{\Delta;\Gamma \vdash e_1 \; e_2 : \tau_2} \quad (\rightarrow_E) \and
  \infer{\Delta, X; \Gamma \vdash e : \tau}{\Delta;\Gamma \vdash \Lambda X.\, e : \forall X.\;\tau} \quad (\forall_I) \and
  \infer{\Delta;\Gamma \vdash e : \forall X.\;\tau}{\Delta;\Gamma \vdash e[\tau'] : \tau[X\mapsto \tau']} \quad (\forall_E) \and
  \infer{\Delta; \Gamma \vdash e : \tau[ X \mapsto \mu X. \; \tau]}{\Delta;\Gamma \vdash roll \, e : \mu X.\;\tau} \quad (\mu_I) \and
  \infer{\Delta;\Gamma \vdash e : \mu X.\;\tau}{\Delta;\Gamma \vdash unroll \, e : \tau[X\mapsto \mu X.\;\tau]} \quad (\mu_E)
\end{mathpar}


\section{Logical relations}

Now, we are no longer in a place where everything is terminating. Below is a naive logical relation.


 \[\V_\rho(\mu X.\; \tau)(v_1 \sim v_2) \triangleq \exists\; v_1', v_2'.\; v_1 = roll\; v_1' \land v_2 = roll\; v_2' \land  \V_\rho(\tau[X \mapsto \mu X.\;\tau]) (v_1' \sim v_2') \]


The issue with the above relation is that it is not well founded. The type is not shrinking.

We will introduce a step index to solve this problem, similar to fuel in Coq.
\newline
So, the logical relations can be defined as follows:\newline
Here 0 represents a state where we have given up expanding and say we can't tell apart the program
\begin{align*}
  \V_\rho(\tau\; @\; 0)(v_1\sim v_2) &\triangleq \textbf{true} 
  \\ 
  \V_\rho(X\; @\; n)(v_1\sim v_2) &\triangleq \rho(X)(v_1,v_2)\\
  \V_\rho(\Unit\; @\; n)(v_1\sim v_2) &\triangleq v_1=v_2=\unit\\
  \V_\rho(\forall X.\;\tau\; @\; n)(v_1\sim v_2) &\triangleq \exists\; e_1,e_2.\;v_1=\Lambda X.\;e_1\land v_2=\Lambda X.\;e_2\land \forall (R\text{ compatible}) , \, \E_{\rho[X\mapsto R]}(\tau\; @\; n)(e_1 \sim e_2)\\
  \V_\rho(\mu X.\;\tau\; @\; n)(v_1\sim v_2) &\triangleq \forall\; n'<n.\; \exists\; v_1', v_2'.\; v_1 = roll\; v_1' \land v_2 = roll\; v_2' \land  \V_\rho(\tau[X \mapsto \mu X.\;\tau \; @ \; n]) (v_1' \sim v_2')\\
  \noalign{\smallskip} \hline \noalign{\smallskip}
  \E_\rho(\tau\;@\;n)(e_1 \sim e_2) &\triangleq \forall \; v_1,v_2.\;e_1\downarrow v_1\land e_2\downarrow v_2\to \V_\rho(\tau\; @ \; n)(v_1\sim v_2)
\end{align*}



\begin{defn}[Good relation context]

  \[\Delta\vDash \rho \xLeftrightarrow{\triangle} \Delta\subseteq \operatorname{dom}(\rho) \]
\end{defn}

\begin{defn}[Semantically related expression substitution with step indices]
  Two parallel substitutions $\gamma_1,\gamma_2$ (partial maps from expression variables to expressions) are semantically related w.r.t. $\Delta ,\Gamma$ and a relation substitution $\rho$ when:
  \[\Delta;\Gamma\vDash \gamma_1\sim \gamma_2 \xLeftrightarrow{\triangle} \forall \rho. \; \Delta \vDash_\rho \to \forall\; x: \tau \in \Gamma. \; \forall\; n.\; \V_\rho(\tau\;@\;n)(\gamma_1(x)\sim \gamma_2(x))\]
\end{defn}

\begin{defn}[Semantically related (in type $\tau$ with step indices)]
  \[\Delta;\Gamma\vDash e_1\sim e_2:\tau \xLeftrightarrow{\triangle} \forall \rho. \; \Delta \vDash_\rho \to \forall\; \gamma_1, \gamma_2. \; \Delta; \Gamma \vDash \gamma_1, \gamma_2 \to \; \forall\; n.\; \E_\rho(\tau\;@\;n)(e_1\sim e_2)\]
\end{defn}

\begin{lem} [Monotonicity]
\[\forall\; n, n'<n, e_1, e_2, \tau, \rho. \; \E_\rho (\tau\;@\;n')(e_1\sim e_2) \]
This means we can give up sooner, i.e. say sooner that the programs are indistinguishable.
\end{lem}
\begin{thm}[Fundamental/Parametricity]
  \[If \; \Delta;\Gamma\vdash e:\tau\; then \; \Delta ; \Gamma \vDash e \sim e:\tau\]
  This reads ``well-typed expressions are related to itself''
\end{thm}
\begin{lem}
  \[If \; \E_{\rho[X \mapsto \V_\rho (\tau\;@\;n)]}(\sigma\;@\;n) (e_1 \sim e_2)\; then \; \E_\rho (\sigma[X \mapsto \tau]\;@\;n) (e_1[X \mapsto \tau] \sim e_2[X \mapsto \tau]) \]
\end{lem}
\section{Homework}
Prove the Theorem and Lemmas mentioned above

\end{document}

%%% Local Variables:
%%% mode: latex
%%% TeX-master: t
%%% End: