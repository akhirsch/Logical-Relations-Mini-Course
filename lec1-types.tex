\documentclass{notes}
\coursename{Logical Relations Mini-Course}
\lecnumber{1}
\title{Type System Fundamentals}
\author{Michael Piskozub}

\begin{document}
\maketitle

\newcommand{\proves}{\vdash}

\newcommand{\fst}[1]{\text{fst} \: #1}
\newcommand{\snd}[1]{\text{snd} \: #1}
\newcommand{\inl}[1]{\text{inl} \: #1}
\newcommand{\inr}[1]{\text{inr} \: #1}
\newcommand{\case}[3]{\text{case} \: (#1, x . #2, y . #3)}
\newcommand{\unit}{()}
\newcommand{\matchdone}[1]{\text{match} \: #1 \: \text{done}}

\newcommand{\betaequiv}{\equiv_{\beta}}
\newcommand{\etaequiv}{\equiv_{\eta}}

\newcommand{\RuleTemplate}[5]{
  \ensuremath{%
    \infer[#1]{#3}{#4}%
%    \if\relax\detokenize{#1}\relax
%      \infer{#3}{#4}
%    \else
%      {\textsc{\rulefiguresize[#1]}}~\infer{#3}{#4}
%    \fi%
  }%
  \if\relax\detokenize{#2}\relax\else\label{#2}\fi%
  \if\relax\detokenize{#5}\relax\else~\text{#5}\fi
}

\newcommand{\Rule}[4][]{%
  \if\relax\detokenize{#2}\relax
    \RuleTemplate{}{}{#3}{#4}{#1}
  \else
    \RuleTemplate{#2}{rule:#2}{#3}{#4}{#1}
  \fi}
\newcommand{\NlRule}[4][]{\RuleTemplate{#2}{}{#3}{#4}{#1}}

\begin{syntax}
  \category[Expressions]{e}
  % Each alternative is introduced on its own line.
  \alternative{x}
  \alternative{\lambda x . e}
  \alternative{e_1 e_2}
  \alternative{(e_1, e_2)}
  \alternative{\fst e}
  \alternative{\snd e}
  \alternative{\inl e}
  \alternative{\inr e}
  \alternative{\case{e_1}{e_2}{e_3}}
  \alternative{\unit}
  \alternative{\matchdone e}
\end{syntax}

\vspace{0.3in}

\bf{Typing Rules:}
\begin{mathpar}
  \Rule{Axiom}{x : \tau \in \Gamma}{\Gamma \proves x : \tau}
  \and
  \Rule{arrow - I}{
    \Gamma, x : \tau \proves e : \sigma
  }{
    \Gamma \proves \lambda x . e : \tau \rightarrow \sigma
  }
  \and
  \Rule{arrow - E}{
    \Gamma \proves e_1 : \tau \rightarrow \sigma \\ \Gamma \proves e_2 : \tau
  }{
    \Gamma \proves e_1 e_2 : \sigma
  }
  \and
  \\
  \Rule{x - I}{
    \Gamma \proves e_1 : \tau
    \\
    \Gamma \proves e_2 : \sigma
  }{
    \Gamma \proves (e_1, e_2) : \tau \times \sigma
  }
  \and
  \Rule{x - E1}{
    \Gamma \proves e : \tau \times \sigma
  }{
    \Gamma \proves \fst{e} : \tau
  }
  \and
  \Rule{x - E2}{
    \Gamma \proves e : \tau \times \sigma
  }{
    \Gamma \proves \snd{e} : \sigma
  }
  \and
  \\
  \Rule{+ - I1}{
    \Gamma \proves e : \tau
  }{
    \Gamma \proves \inl{e} : \tau + \sigma
  }
  \and
  \Rule{+ - I2}{
    \Gamma \proves e : \sigma
  }{
    \Gamma \proves \inr{e} : \tau + \sigma
  }
  \and
  \Rule{+ - E}{
    \Gamma \proves e_1 : \tau + \sigma
    \\
    \Gamma, x : \tau \proves e_2 : \rho
    \\
    \Gamma, y : \sigma \proves e_3 : \rho
  }{
    \Gamma \proves \case{e_1}{e_2}{e_3} : \rho
  }
  \and
  \\
  \Rule{unit - I}{ }{
    \Gamma \proves \unit : unit
  }
  \and
  \Rule{void - E}{
    \Gamma \proves e : void
  }{
    \Gamma \proves \matchdone{e} : \tau
  }
\end{mathpar}
\begin{nb}
  There is a duality of types here. Products are duals of coproducts and $void$ is
  the dual of $unit$. For any dual types, the introduction forms of one correspond 
  to the elimination forms of the other, and vice versa.
\end{nb}

\vspace{0.3in}

\bf{$\beta$ and $\eta$ Equivalence:}

\begin{nb}
  $\beta$ and $\eta$ equivalences come directly from introduction and
  elimination rules.
  \\
  $\beta$ equivalence: Apply elimination directly to an introduction form.
  \\
  $\eta$ equivalence: Apply introduction directly to an elimination form.
\end{nb}

\begin{mathpar}
  (\lambda x . e_1) e_2 \betaequiv e_1[x \mapsto e_2]
  \and
  f \etaequiv \lambda x . (f \: x)

  \and
  \\
  \fst (e_1, e_2) \betaequiv e_1
  \and
  \snd (e_1, e_2) \betaequiv e_2
  \and
  (\fst{e}, \snd{e}) \etaequiv e
\end{mathpar}

\begin{nb}
  Since product types have two elimination rules, there are two $\beta$ equivalences.
  Likewise, since product types have one introduction rule, there is one $\eta$ rule.
\end{nb}

\begin{mathpar}
  \case{\inl{e_1}}{e_2}{e_3} \betaequiv e_2[x \mapsto e_1]
  \and
  \case{\inr{e_1}}{e_2}{e_3} \betaequiv e_3[y \mapsto e_1]
  \and
  \\
  \case{e_1}{\inl{x}}{\inr{y}} \etaequiv e_1
\end{mathpar}

\begin{nb}
  Note that the continuation of a case expression is $embedded$ in the case
  expression. So, the introduction rules must be embedded in the continuations
  of the case expression in the $\eta$ equivalence.
\end{nb}

\begin{mathpar}
  e \etaequiv \unit
  \and
  \text{C}[\matchdone{e}] \betaequiv \matchdone{e}
\end{mathpar}

\begin{nb}
  The $\eta$ rule says that if you have $e : unit$, then that it is $\eta$
  equivalent to the introduction form.
  \\
  In the $\beta$ rule, $\bf{C}[.]$ is an evaluation context. Something of type 
  $void$ cannot be built, so if there is an expression $e : void$ in your
  computation, we can skip the entire computation and just give back 
  $\bf{match}$ $e$ $\bf{done}$.
\end{nb}

\end{document}