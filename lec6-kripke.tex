\documentclass{notes}
\coursename{Logical Relations Mini-Course}
\lecnumber{6}
\title{Kripke Logical Relations}
\author{Michael Piskozub}

\begin{document}
\maketitle

\newcommand{\proves}{\vdash}

% Expression commands
\newcommand{\new}[2]{\textsf{new}\langle#1\rangle(#2)}
\newcommand*{\defeq}{\mathrel{\vcenter{\baselineskip0.5ex \lineskiplimit0pt
                     \hbox{\scriptsize.}\hbox{\scriptsize.}}}%
                     =}
\newcommand{\store}[2]{#1 \defeq #2}
\newcommand{\deref}[1]{! \, #1}

% Type commands
\newcommand{\bool}{\textsf{bool}}
\newcommand{\prodtyp}[2]{#1 \: \times \: #2}
\newcommand{\sumtyp}[2]{#1 \: + \: #2}
\newcommand{\arrow}[2]{#1 \: \rightarrow \: #2}
\newcommand{\unit}{\textsf{unit}}
\newcommand{\reftyp}[1]{\textsf{ref} \:\: #1}

% Typing judgement
\newcommand{\typing}[5]{#1; #2 \proves #3 : #4 \dashv #5}

% Step
\newcommand{\step}[4]{#1; #2 \longrightarrow #3; #4}

% Logical relation
\newcommand{\Vold}[2]{V(#1)(#2)}
\newcommand{\Vnew}[3]{V(#1)(#2 \; @ \; #3)}
\newcommand{\Eold}[2]{E(#1)(#2)}
\newcommand{\Enew}[3]{E(#1)(#2 \; @ \; #3)}

% Misc
\newcommand{\dom}[1]{\textsf{dom}(#1)}
\newcommand{\deq}{\overset{\Delta}{=}}
\newcommand{\defas}[2]{#1 \; \deq \; #2}
\newcommand{\deqv}{\overset{\Delta}{\iff}}
\newcommand{\defiff}[2]{#1 \; \deqv \; #2}

\section{Introduction}

In today's lecture, we discussed a style of logical relations called \emph{Kripke Logical Relations}.
Kripke Logical Relations allow us to model state, which allows us to model many types of effects.

\section{Our Language}

In our language, we do not have recursive types or recursive functions, but by virtue of having state, we can write programs that do not terminate.

\begin{syntax}
  \category[Expressions]{e}
  % Each alternative is introduced on its own line.
  \alternative{...}
  \alternative{\ell}
  \alternative{\new{\tau}{e}}
  \alternative{\store{e}{e}}
  \alternative{\deref{e}}

  % You can add vertical spacing between categories to visually group them.
  % \separate
  % You can pass the amount of space explicitly if you want to manually control it:
  % \separate[5ex]

  \category[Values]{v}
  \alternative{...}
  \alternative{\ell}

  \category[Types]{\tau}
  \alternative{\bool}
  \alternative{\prodtyp{\tau_1}{\tau_2}}
  \alternative{\sumtyp{\tau_1}{\tau_2}}
  \alternative{\arrow{\tau_1}{\tau_2}}
  \alternative{\unit}
  \alternative{\reftyp{\tau}}
\end{syntax}

\vspace{0.1in}

Typing judgements have the form: $\typing{\Gamma}{\Theta}{e}{\tau}{\Theta'}$, where $\Theta$ maps locations to types.

\begin{mathpar}
  \infer[T-Loc]{\ell : \tau \in \Theta}{\typing{\Gamma}{\Theta}{\ell}{\reftyp{\tau}}{\Theta'}} \and
  \infer[T-New]{
    \typing{\Gamma}{\Theta}{e}{\tau}{\Theta'}
  }{
    \typing{\Gamma}{\Theta}{\new{\tau}{e}}{\reftyp{\tau}}{\Theta', \ell : \tau}
  } \and
  \infer[T-Store]{
    \typing{\Gamma}{\Theta_1}{e_1}{\reftyp{\tau}}{\Theta_2} \\
    \typing{\Gamma}{\Theta_2}{e_2}{\tau}{\Theta_3}
  }{
    \typing{\Gamma}{\Theta_1}{\store{e_1}{e_2}}{\unit}{\Theta_3}
  } \and
  \infer[T-Deref]{
    \typing{\Gamma}{\Theta}{e}{\reftyp{\tau}}{\Theta'}
  }{
    \typing{\Gamma}{\Theta}{\deref{e}}{\tau}{\Theta'}
  }
\end{mathpar}

The operational semantics of our language are defined as follows.
Note that $\theta$ maps locations to values.

\begin{mathpar}
  \infer[S-New]{
    \ell \notin \dom{\theta}
  }{
    \step{\theta}{\new{\tau}{v}}{\theta}{\ell \mapsto v; \ell}
  } \and
  \infer[S-Store]{
    \ell \in \dom{\theta}
  }{
    \step{\theta}{\store{\ell}{v}}{\theta, \ell \mapsto v}{()}
  } \and
  \infer[S-Deref]{
    \theta(\ell) = v
  }{
    \step{\theta}{\deref{\ell}}{\theta}{v}
  }
\end{mathpar}

Note that we can model recursive functions using state.
Hence, we will have to deal with expressions that do not terminate.

For example, consider the following program:
\begin{quote}
  \verb|let f = new<int->int>(|$\lambda$\verb|x.x) in| \\
  \verb|let g = |$\lambda$\verb|x.(!f) x in| \\
  \verb|f := g; g 42|
\end{quote}

This program does not terminate because the call to \verb|g| keeps calling itself recursively forever.

\section{First Attempt at a Logical Relation}

A first attempt at a logical relations model for our language might look as follows. \\ \\
First, the value relation:

\begin{mathpar}
  \defas{\Vold{\reftyp{\tau}}{v}}{\exists \ell . \; v = \ell} \\
\end{mathpar}
This says that $v$ is a location, but we do not make any guarantees about what is at that location. \\ \\
The expression relation might look like:

\begin{mathpar}
  \defas{\Eold{\tau}{e}}{\forall v . \; e \downarrow v \Rightarrow \Vold{\tau}{v}}
\end{mathpar}
However, we are not given any information about the state at expression $e$.
Hence, we need something more for our logical relation.

\section{Kripke Logical Relation}

Saul Kripke was interested in the question: what does it mean to know something?
Mathematicians and philosophers already had a good idea of what it meant to know whether something is true in some model, but did not know how to model knowledge.
Kripke modeled knowledge in terms of \emph{worlds}.

For example, consider someone, call him Andrew, knows that it is raining in Buffalo, New York, but knows nothing about the weather in Madison, Wisconsin.
A world in which it is raining in Buffalo and sunny in Madison, versus a world in which it is raining in Buffalo and raining in Madison are equally plausible to Andrew.
Andrew does not know which of those two worlds he is in.

Kripke modeled knowledge as follows.
If you have an equivalence class of worlds that you believe in, and something is true in all of those worlds, then you know it. \\

\subsection{Kripke Worlds}

We introduce the following new notation: $W \leq W'$.
Informally, this says: "If I'm in world $W$, then $W'$ can be a possible future world."

However, we still have not defined what a world actually is.
It turns out that the notion of a world is different for unary vs. binary logical relations.

\subsubsection{Unary Case}

Define world as $\theta$. \\

$\defiff{\theta \leq \theta'}{\theta \subseteq \theta'}$

\subsubsection{Binary Case}

A world is a triple $\langle \theta_1, \theta_2, \cong \rangle$ where $\cong$ is the isomorphism $\dom{\theta_1} \leftrightarrow \dom{\theta_2}$. \\

$\defiff{W \leq W}{
  (W.\theta_1 \subseteq W'.\theta_1) \; \land \; (W.\theta_2 \subseteq W'.\theta_2) \; \land \; (W.\cong \subseteq W'.\cong)
}$

\subsubsection{Unary Logical Relation}

We can now define the unary logical relation.

\begin{equation}
  \defas{
    \Vnew{\reftyp{\tau}}{v}{\theta}
  }{
    \exists \ell . \; v = \ell \; \land \; \Vold{\tau}{\theta(\ell)}
  }
\end{equation}

\begin{equation}
  \defas{
    \Vnew{\arrow{\tau}{\sigma}}{v}{\theta}
  }{
    \exists x, e . \; v = \lambda x . e \; \land \; (\forall \theta' \geq \theta . \forall v' . \; \Vnew{\tau}{v'}{\theta} \Rightarrow \Enew{\sigma}{e[x \mapsto v]}{\theta'})
  }
\end{equation}

\begin{nb}
  The value relation on the arrow type is required or else we get stuck in the application case of the fundamental theorem.
\end{nb}

\begin{equation}
  \begin{aligned}
    \Enew{\tau}{e}{\theta} \deq \; & ((\exists \theta' , e' . \; \theta; e \longrightarrow \theta'; e') \\
    & \land \; (\forall \theta', e' . \; \theta; e \longrightarrow \theta'; e' \Rightarrow \theta \leq \theta' \; \land \; \Enew{\tau}{e'}{\theta'})) \\
    & \lor \; \Vnew{\tau}{e}{\theta}
  \end{aligned}
\end{equation}

The expression relation above can be equivalently defined as:

\begin{equation}
  \begin{aligned}
    \Enew{\tau}{e}{\theta} \deq \; & ((\exists \theta' , v . \; \theta; e \downarrow \theta'; v) \\
    & \land \; (\forall \theta', v . \; \theta; e \downarrow \theta'; v \Rightarrow \Vnew{\tau}{e}{\theta})) \\
    & \lor \; \theta;e \uparrow
  \end{aligned}
\end{equation}

\subsubsection{Binary Logical Relation}

The binary logical relation can be defined as follows.

\begin{equation}
  \begin{aligned}
    \Vnew{\arrow{\tau}{\sigma}}{v \sim v'}{W} \deq \; & \exists x, y, e_1, e_2 . \; v = \lambda x . e_1 \; \land \; v' = \lambda y . e_2 \\
    & \land \forall W' \geq W. \forall v_1, v_2. \Vnew{\tau}{v_1 \sim v_2}{W} \Rightarrow \Enew{\sigma}{e_1[x \mapsto v_1] \sim e_2[y \mapsto v_2]}{W'}
  \end{aligned}
\end{equation}

\begin{equation}
  \begin{aligned}
    \Vnew{\reftyp{\tau}}{v_1 \sim v_2}{W} \deq \; & \exists \ell_1, \ell_2 . \; \ell_1 \; W.\cong \; \ell_2 \\
    & \land \; \Vnew{\tau}{W.\theta_1(\ell_1) \sim W.\theta_2(\ell_2)}{W}
  \end{aligned}
\end{equation}

\begin{equation}
  \begin{aligned}
    \Enew{\tau}{e_1 \sim e_2}{W} \deq \; & \Vnew{\tau}{e_1 \sim e_2}{W} \\
    & \lor \; (
      (\exists e_1', e_2', W' \geq W . \; W.\theta_1; e_1 \longrightarrow W'.\theta_1; e_1' \; \land \; W.\theta_2; e_2 \longrightarrow W'.\theta_2; e'_2) \\
      & \; \; \; \; \; \; \; \land \; (\forall e'_1, e'_2, W' \geq W . \; W.\theta_1; e_1 \longrightarrow W'.\theta_1; e'_1 \; \land \; W.\theta_2; e_2 \longrightarrow W'.\theta_2 e'_2 \Rightarrow \Enew{\tau}{e'_1 \sim e'_2}{W'})
    )
  \end{aligned}
\end{equation}

\vspace{0.1in}

However, the above definition of the expression relation is quite strong.
For instance, \verb|(|$\lambda$\verb|x.42) 5|\;$\sim$\;\verb|42| is not in the above expression relation. \\ \\
A more flexible variant of the expression relation is as follows:

\begin{equation}
  \begin{aligned}
    E'(\tau)(e_1 \sim e_2 \; @ \; W) \deq \; & (
      (\exists v_1', v_2', W' \geq W . \; W.\theta_1; e_1 \downarrow W'.\theta_2; v_1' \; \land \; W.\theta_2; e_2 \downarrow W'.\theta_2; v'_2) \\
      & \land \; (\forall v'_1, v'_2, W' \geq W . \; W.\theta_1; e_1 \downarrow W'.\theta_1; v'_1 \; \land \; W.\theta_2; e_2 \downarrow W'.\theta_2 v'_2 \Rightarrow \Vnew{\tau}{v'_1 \sim v'_2}{W'})
    ) \\
    & \lor \; (W.\theta_1; e_1 \uparrow \; \land \; W.\theta_2; e_2 \uparrow)
  \end{aligned}
\end{equation}

In this case, \verb|(|$\lambda$\verb|x.42) 5|\;$\sim$\;\verb|42| is in the expression relation $E'$.

\end{document}